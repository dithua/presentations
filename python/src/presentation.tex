\documentclass{bredelebeamer}
\usepackage{hyperref}
\usepackage{listings}


\lstdefinestyle{customc}{
  belowcaptionskip=1\baselineskip,
  breaklines=true,
  language=Python,
  keepspaces=true,
  columns=flexible,
  showstringspaces=false,
  basicstyle=\footnotesize\ttfamily,
  keywordstyle=\bfseries\color{purple!90!black},
  commentstyle=\itshape\color{gray!40!black},
  identifierstyle=\color{blue!70},
  stringstyle=\color{orange},
}

\lstset{escapechar=@,style=customc}

%%%%%%%%%%%%%%%%%%%%%%%%%%%%%%%%%%%%%%%%%%%%%%%%



\title[Exploring python]{Exploring python with a background in Java}
% Titre du diaporama

% Sous-titre optionnel

\author{Andreas Grivas\inst{1}}
% La commande \inst{...} Permet d'afficher l' affiliation de l'intervenant.
% Si il y a plusieurs intervenants: Marcel Dupont\inst{1}, Roger Durand\inst{2}
% Il suffit alors d'ajouter un autre institut sur le modèle ci-dessous.

\institute[Harokopeio University of Athens]
{
  \inst{1}%
  Informatics and Telematics\\
  Harokopeio University of Athens
  }


\date{\today}
% Optionnel. La date, généralement celle du jour de la conférence

\subject{Sujet de votre diaporama}
% C'est utilisé dans les métadonnes du PDF



\logo{
\includegraphics[scale=0.08]{images/py.png}
}



%%%%%%%%%%%%%%%%%%%%%%%%%%%%%%%%%%%%%%%%%%%%%%%%%%%%%%%%%%%%%%%%%%%%%
\begin{document}

\begin{frame}
  \titlepage
\end{frame}





\begin{frame}{Summary}
  \tableofcontents
  % possibilité d'ajouter l'option [pausesections]
\end{frame}




\section{Who are you and why are we here?}

\begin{frame}{Disclaimers}
  \begin{block}{self.disclaimers}
  \begin{itemize}
  \item No expertise in python is implied
  \item Approaches reflect personal taste and may or may not be \href{https://github.com/python/cpython/blob/c7688b44387d116522ff53c0927169db45969f0e/Lib/this.py}{\emph{pythonic}}
  \item Will cover features I consider useful
  \end{itemize}
  \end{block}
\end{frame}

\begin{frame}{We are here ..}

\begin{columns}

\begin{column}{0.5\textwidth}

\begin{exampleblock}{to}
\begin{itemize}
\item Understand the \emph{basics}
\item See \emph{practical} python examples
\item Stress \emph{philosophy}
\end{itemize}
\end{exampleblock}

\end{column}

\begin{column}{0.5\textwidth}
\begin{alertblock}{and not to}
\begin{itemize}
\item Become technical
\item Compare python libraries and tools
\end{itemize}
\end{alertblock}
\end{column}

\end{columns}
\end{frame}

\section{What is Python?}

\begin{frame}[fragile]{Python info}

  \begin{lstlisting}
  # incompatibility example
  print 'hello world'  # python 2
  print('hello world') # python 3
  \end{lstlisting}

  \begin{itemize}
    \item A laconic programming language that emphasizes readability
    \item Implementation started in December 1989 by Guido van Rossum
    \item Written in C
    \item Current stable versions 2.7.9 \& 3.4.3
    \item Python3 is backwards incompatible with Python2
    \item \href{https://github.com/python-git/python}{\emph{Open source - https://github.com/python-git/python}}
  \end{itemize}


\end{frame}


\begin{frame}[fragile]{Python is..}

\begin{lstlisting}
# hello python in line comments!

message = 'hello python'   # 1. I didn't say String message
print(message + 5)         # 2. Error!
print(message + str(5))    # 2. Ok.
print('num :' + str(5))    # 3. I could run this line of code alone
def my_favourite_number(): # 4. I can define a function not in a class
    return -32
class FavouriteNumber:     # 4. I could use a class
    num = -32
\end{lstlisting}

\begin{enumerate}
  \item \emph{Dynamically typed} - no variable types (\exemple{think javascript})
  \item \emph{Strongly typed} - no guessing if not well defined (\exemple{don't think javascript})
  \item Feels \emph{interpreted} - can run one line of code \footnote{Depends on implementation of python (CPython, Jython, IronPython, Stackless}(\exemple{think octave})
  \item Supports many \emph{programming paradigms} (\exemple{don't think Java})
  \begin{enumerate}
    \item \emph{Procedural} - your building blocks are functions (\exemple{think C})
    \item \emph{ObjectOriented} - your building blocks are classes (\exemple{think Java})
    \item More..
  \end{enumerate}
\end{enumerate}

\end{frame}

\begin{frame}{Philosophy}
  Python is a loose, programmer friendly language that is based on good intentions. \\
  It supposes you will play nicely with other programmers and use your freedoms wisely.\\
  It doesn't impose restrictions supposing that you are intelligent enough not to shoot
  your own foot just because you were given a gun.\\
  
  \begin{block}{Why do you say that?}
  \begin{itemize}
    \item Use programming paradigm that matches the situation
    \item Don't have to capture exceptions
    \item Operator overloading
    \item No encapsulation
  \end{itemize}
  \end{block}
\end{frame}

\section{Write some code}

\begin{frame}{IPython}
We will be exploring python with ipython, python3 and an editor if needed.\\

  \begin{block}{Useful links}
    \begin{itemize}
      \item \href{http://www.pythontutor.com/visualize.html\#mode=edit}{\emph{Python Tutor - visualize code execution}}
      \item \href{http://ipython.org/notebook.html}{\emph{IPython notebooks - check these out}}
      \item \href{http://ipython.org/}{\emph{IPython website}}
    \end{itemize}
  \end{block}
\end{frame}

\subsection{Basics}
\begin{frame}[fragile]{Variables}
  \begin{columns}
    \begin{column}{0.5\textwidth}
      \begin{exampleblock}{definition examples}
        \begin{lstlisting}
    # Basic variables example 
    # with terrible choices for names.
    # as we said before - no types

    # basic types
    b = True    # bool
    i = 5       # int
    f = 5.0     # float
    s = 'hi'    # str

    # built in data structures
    # iterables
    t = ()      # empty tuple
    l = []      # empty list
    d = {}      # empty dictionary
    s = set()   # empty set
        \end{lstlisting}
      \end{exampleblock}
    \end{column}

    \begin{column}{0.5\textwidth}
      \begin{exampleblock}{naming examples}
        \begin{lstlisting}
    # Naming examples 
    # More important to have good names
    # because of no type

    # variable Java style - CamelCase
    randomInteger = 5

    # variable C style
    random_integer = 5

    # classes start uppercase
    class BeerGenerator:
        pass

    # functions underscored with lowercase
    def spawn_food(location):
        pass
        \end{lstlisting}
      \end{exampleblock}
    \end{column}
  \end{columns}
\end{frame}

\begin{frame}[fragile]{Printing}
  \begin{exampleblock}{3 ways to print}
    \begin{lstlisting}
    # I know 3 ways
    
    name = 'Jeremiah'
    short_pie = 3.1415

    # C style formatted
    print('hello %s, have some %.2f' % (name, short_pie))

    # use format
    print('hello {}, have some {}'.format(name, short_pie))

    # Java style using + operator
    print('hello ' + name + ', have some ' + str(short_pie))
    \end{lstlisting}
  \end{exampleblock}
\end{frame}

\begin{frame}[fragile]{Imports}
  \begin{exampleblock}{using libraries}
    \begin{lstlisting}
    # use a python library
    # below examples are equivalent
    
    # include the random module in namespace
    import random
    rand = random.randrange(10)
    
    # include random with alias
    import random as rnd 
    rand = rnd.randrange(10)
    
    # include target object from module
    from random import randrange
    rand = randrange(10)
    
    # include target object only with alias
    from random import randrange as rrange
    rand = rrange(10)
    \end{lstlisting}
  \end{exampleblock}
\end{frame}

\begin{frame}[fragile]{Example}
  \begin{exampleblock}{script}
    \begin{lstlisting}
    # 1st script
    from random import randrange     # randrange returns a random number 0 - to target - 1
    dice_roll = randrange(6) + 1     # get a dice roll - could have done it: randrange(1,7)
    character_roll = randrange(26)   # 26 letters in the alphabet - pick a value out of 26
    ascii_a_value = ord('a')         # get ASCII code of a
    print(type(ascii_a_value))       # int
    rand_char = chr(ascii_a_value + character_roll)  # pick a random char
    print(type(rand_char))           # str - no char
    message = rand_char * dice_roll  # what message is this? multiplication on string?
    print(message)                   # print it!
    \end{lstlisting}
  \end{exampleblock}
\end{frame}

\subsection{Flow - Control}
\begin{frame}[fragile]{If Else}
  \begin{exampleblock}{what if..}
    \begin{lstlisting}
    # Decisions.. decisions..
    # Goodbye curly braces {}!
    # Indentation Warning - it matters!
    
    using_tabs = True
    if using_tabs:
        print('You are going to have a tough time -_-')
    else:
        print('Whatever - it\'s not going to print')
        
    # bonus - if you are just going to initialize something
    programmer = 'Happy' if not using_tabs else 'Unhappy'
    \end{lstlisting}
  \end{exampleblock}
\end{frame}

\begin{frame}[fragile]{For - While}
  \begin{exampleblock}{for.. starters}
    \begin{lstlisting}
    # for.. is boring like this - why later
    # so.. here's a boring example
    
    n = 10
    for x in range(n):
       print(x)
       
    print()           # leave a newline
    
    x = n
    # while
    while(x>0):
        x -= 1        # sorry C guys - no x--
        if(x%2==0):
            continue  # don't print even numbers
        print(x)
    \end{lstlisting}
  \end{exampleblock}
\end{frame}

\begin{frame}[fragile]{Exceptions}
  \begin{exampleblock}{try catch..}
    \begin{lstlisting}
    # Ah.. exceptions
    # Note to java programmers!
    # It's error not exception
    # It's except and not catch
    # It's raise not throw
    # And don't overuse this feature
    
    try:
        print('This gotcha will happen at least ' + 10 + ' more times')
    except TypeError as e:
        print('I can\'t do that')
        print(e)
    \end{lstlisting}
  \end{exampleblock}
\end{frame}

\begin{frame}[fragile]{With}
  \begin{exampleblock}{with .. as}
    \begin{lstlisting}
    # Safely exit construct - like a finally
    # Even if an error occurs file will be closed
    # Catching exception is meaningful if you can do something
    # or you want to print a nicer message
    
    filename = 'examples.txt'
    with open(filename, 'r') as f:
        lines = f.readlines()
    \end{lstlisting}
  \end{exampleblock}
\end{frame}

\subsection{Structures}

\begin{frame}[fragile]{List}
  \begin{exampleblock}{shopping..}
    \begin{lstlisting}
    # List is iterable - ordered
    # List is mutable
    
    shopping = list('spam', 'flour', 'spam', 'eggs')  # create a list
    
    shopping = ['spam', 'flour', 'spam', 'eggs']      # equivalent
    
    num_items = len(shopping)  # get size of list
    print(shopping)            # print whole list
    print(shopping[0])         # access first element and print it
    shopping[0] = 'cookies'    # change first element
    print(shopping[0])         # access changed element and print
    
    # much better for example - no indexes
    for item in shopping:
        print(item)
        
    # if you really really need indexes
    for index, item in enumerate(shopping):
        print('item %s in list: %s' % (index, item))
        
    # list comprehension <3
    no_spam = [item for item in shopping if item != 'spam']
    \end{lstlisting}
  \end{exampleblock}
\end{frame}

\begin{frame}[fragile]{Tuple}
  \begin{exampleblock}{unchangeable list}
    \begin{lstlisting}
    # Tuple is iterable - ordered
    # Tuple is immutable - can't change it after initialization
    
    shopping = tuple('spam', 'flour', 'spam', 'eggs')  # create a tuple
     
    shopping = ('spam', 'flour', 'spam', 'eggs')       # equivalent
    
    num_items = len(shopping)  # get size of tuple
    print(shopping)            # print whole tuple
    print(shopping[0])         # access first element and print it
    shopping[0] = 'cookies'    # oops.. no can do
    print(shopping[0])         # well.. the same as before
    
    # previous example also works - show reversed - these functions can be chained
    for item in reversed(shopping):
        print(item)
        
    # if you really really need indexes
    for index, item in enumerate(shopping):
        print('item %s in list: %s' % (index, item))
        
    # tuple comprehension <3
    no_spam = [item for item in shopping if item != 'spam']
    \end{lstlisting}
  \end{exampleblock}
\end{frame}

\begin{frame}[fragile]{Set}
  \begin{exampleblock}{no duplicates}
    \begin{lstlisting}
    # Set is iterable - unordered (can't access using [])
    # Set is mutable
    
    shopping = set('spam', 'flour', 'spam', 'eggs')  # create a set
    
    shopping = {'spam', 'flour', 'spam', 'eggs'}     # equivalent
    
    num_items = len(shopping)  # get size of set
    print(shopping)            # print whole set
    print(shopping[0])         # nope. set is unordered - no meaning in first element
    print('spam' in shopping)  # with sets we usually check for membership
    shopping[0] = 'cookies'    # oops.. no can do
    
    # previous example also works - show reversed - these functions can be chained
    for item in shopping:
        print(item)
        
    # set comprehension <3
    no_spam = [item for item in shopping if item != 'spam']
    \end{lstlisting}
  \end{exampleblock}
\end{frame}

\begin{frame}[fragile]{Dictionary}
  \begin{exampleblock}{no duplicates + mapping}
    \begin{lstlisting}
    # Dictionary is iterable - unordered
    # Dictionary is mutable
    
    # create a dictionary - keep amount
    shopping = dict('spam':1, 'flour':1, 'eggs':1)
    
    # equivalent
    shopping = {'spam', 'flour', 'spam', 'eggs'}
    
    num_items = len(shopping)  # get size of set
    print(shopping)            # print whole tuple
    print(shopping[0])         # nope. dictionary is unordered
    print(shopping['spam'])    # yes - this is how we use it
    print('spam' in shopping ) # check f
    shopping['spam'] = 3       # change value - yes it is mutable
    
    # previous example also works - show reversed - these functions can be chained
    for key, value in shopping.items():
        print('%s has value %s', (key, value))
        
    # dict comprehension <3
    no_spam = {key:value for key, value in shopping if key != 'spam'}
    \end{lstlisting}
  \end{exampleblock}
\end{frame}

\begin{frame}[fragile]{Example}
  \begin{exampleblock}{this might feel weird}
    \begin{lstlisting}
    # well - let's see if you supposed
    # all members must be of same type
    
    # fast forward! Ignore this for now
    class Stuff:
        # yeah i know - you're thinking underscores
        def __init__(self, name, is_weird):
            self.name = name
            self.is_weird = is_weird
            
        def __repr__(self):
          return '%s - %s' % (self.name, self.is_weird)
            
    my_stuff = Stuff('hokus', True)
    # applies for all - not only lists
    wtf = ['foo', 5, 3.14, ('hmm..',), my_stuff]
    
    for item in wtf:
        print item
        
    # now is a good time to think
    # why you use inheritance so much in java
    \end{lstlisting}
  \end{exampleblock}
\end{frame}

\begin{frame}[fragile]{Example}
  \begin{exampleblock}{reading csv file}
    \begin{lstlisting}
    # don't need to do it like this - you can use pandas
    # we are also not checking escape sequences etc
    # demonstration of easyness
    
    # create the file - this won't be necessary
    with open('myfile.csv', 'w') as csv:
        csv.write('two,minutes,to,midnight\n')
        csv.write('run,to,the,hills\n')
        
    # read the file we wrote
    with open('myfile.csv', 'r') as csv:
        values = [each.rstrip().split(',') for each in csv]
    print(values)
    \end{lstlisting}
  \end{exampleblock}
\end{frame}

% list tuple set dict
\subsection{Functions}
\begin{frame}[fragile]{Ordered arguments}
  \begin{exampleblock}{by order}
    \begin{lstlisting}
    # Function definitions
    # None is like null
    # returns something if specified else None
    # arguments with order or named
    
    def is_hot(temperature):
        hot = temperature >= 30
        return hot
        
    def is_hot(temperature, threshold):
        hot = temperature >= threshold
        return hot
    
    # default argument - must be after non default arguments
    def is_hot(temperature, threshold=30):
        hot = temperature >= threshold
        return hot
        
    # argument with order
    print(is_hot(31))
    # argument by name - doesn't need to be in order
    print(is_hot(31, 30))
    \end{lstlisting}
  \end{exampleblock}
\end{frame}

\begin{frame}[fragile]{Keyword arguments}
  \begin{exampleblock}{by keyword}
    \begin{lstlisting}
    # Function definitions
    # None is like null
    # returns something if specified else None
    # arguments with order or named
    
    def is_hot(temperature):
        hot = temperature >= 30
        return hot
        
    def is_hot(temperature, threshold):
        hot = temperature >= threshold
        return hot
    
    # default argument - must be after non default arguments
    def is_hot(temperature, threshold=30):
        hot = temperature >= threshold
        return hot
        
    # pass argument by keyword
    print(is_hot(temperature=31))
    # pass argument by keyword - doesn't need to be in order
    print(is_hot(threshold=20, temperature=31))
    \end{lstlisting}
  \end{exampleblock}
\end{frame}

% define a method - default arguments *args **kwargs in footnote
\subsection{Classes}
% define a class

\begin{frame}[fragile]{Class - representation}
  \begin{exampleblock}{hang on in there}
    \begin{lstlisting}
    # make class human readable!
    # class definition
    
    class Animal(object):
        
        # kinda java static
        has_soul = True
        # constructor
        def __init__(self, name):
            self.name = name
        def __repr__(self):
            return '\n'.join(['%s : %s' % (key, value) 
                              for key, value in 
                              self.__dict__.items()]
    
    doug = Animal(name='Doug')
    # this calls repr function
    print(doug)
    # all is going to be ok..
    doug.kind = 'dog'
    doug.age = 6
    print(doug)
    \end{lstlisting}
  \end{exampleblock}
\end{frame}

\section{Closing notes}

\begin{frame}[fragile]{kwargs}
  In case someone wanted to try out kwargs
  \begin{exampleblock}{How to use kwargs}
    \begin{lstlisting}
    # I forgot to unpack the dictionary -_-
    # example function that simply prints the arguments
    def unknown_args(**kwargs):
        for key, val in kwargs.items():
            print('argument %s has value %s' % (key, val))

    # create a dictionary of arguments
    arguments = {'name': 'Ollie', 'age': 5, 'is_cool': True}
    # call the function
    unknown_args(**arguments) #  needs to be called with **
    \end{lstlisting}
  \end{exampleblock}
\end{frame}


\begin{frame}{Don't miss}
  \begin{block}{language features}
  \begin{itemize}
    \item *args and **kwargs in functions
    \item functions: zip, map, filter
    \item decorators
    \item class inheritance
    \item generators
    \item lambda expressions
  \end{itemize}
  \end{block}
  
  \begin{exampleblock}{libraries}
  \begin{itemize}
    \item itertools, collections - python "stdlib"
    \item pandas - data structures
    \item numpy - scipy - number crunching
    \item pyyaml - yaml file format <- good for config files
  \end{itemize}
  \end{exampleblock}
\end{frame}

\begin{frame}[fragile]{The End}
    \begin{lstlisting}
      self.mute()
      print('Any questions?')
    \end{lstlisting}
\end{frame}
\end{document}
